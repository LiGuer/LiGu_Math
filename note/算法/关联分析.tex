* 关联分析
	\Situation
		根据频繁项集, 寻找数据集中变量之间的关联规则.
		- 频繁项集: 经常出现在一块的物品的集合.
		- 关联规则: 两种物品之间可能存在很强的关系.
		- 支持度 $\P(AB)$: 数据集中包含该项集的记录所占的比例. $\P(AB) ≌ num(AB) / num(all)$
		- 置信度 $\P(A→B) = \P(B|A) = \/{\P(AB)}{\P(A)}$

	\Problem

	\Property
		- 项集频繁, 则其子集频繁$ <=> $项集不频繁, 则其超集不频繁.
		- 若规则X→Y−X低于置信度阈值, 则对于X子集X',规则X'→Y−X'也低于置信度阈值
		- 频繁项集生成的方法:
			- $F_k = F_{k-1} × F_1$
			- $F_k = F_{k-1} × F_{k-1}$

	\Algorithm
		- Apriori
			- 输入/输出:
				输入: (1) 初始频繁项集	(2) 最小支持度
				输出: (1) 关联项集		(2) 关联项集支持度

			\Note
				- 频繁项集每一项各不相同,  每一项内部排列有序.

			- 步骤:
				- 频繁项集生成,对于K项的集合
					- 频繁项集子集生成. 生成K项所有可以组合的集合. eg.(frozenset({2, 3}), frozenset({3, 5})) -> (frozenset({2, 3, 5}))
					- 保存满足目标支持度P(AB)的集合.
				-  关联规则生成, 对不同长度(K)的频繁项集依次分析
					- 频繁项集只有两个元素{AB}, 直接计算置信度P(A→B),P(B→A)
					- 频繁项集超过两个元素{ABC...}, 依次计算置信度P(AC...→B)
					- 保存满足目标置信度的关联规则.
