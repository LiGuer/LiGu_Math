* 插值
	\Problem
		求一个函数(曲线)$f(\.x, \.w) = 0$ (或 $y = f(\.x, \.w)$),使得该函数(曲线)经过所有样本点$\{\.x_i\ |\ i = 1:N\}$ (或 $\{(\.x_i, y_i)\ |\  i = 1:N\}$).

	\Algorithm
		- Lagrange插值
			$
				f(x) = \sum_{i=1}^n  y_i · f_i(x)
				f_i(x) = \prod_{j=1,i≠j}^n  \/{x - x_j}{x_i - x_j}
			$
			第N点y = 基函数1 × 第1点y + 基函数2 × 第2点y + 基函数3 × 第3点y
			基函数状态2 = (输入X-第1点x)(输入X-第3点x) / (第2点x-第1点x)(第2点x-第3点x)

		- 样条插值
			通过求解三弯矩方程组得出曲线函数组的过程

		- Kriging插值
			- 普通Kriging插值
				- 目的:
					空间插值. 满足假设:
					- 空间属性z是均一的. 对空间任意一点, 都有相同期望、方差.
					$
						\min \qu&	var = 2\sum w_i γ_{i0} - \sum\sum w_i w_j γ_{ij} - γ_{00}
						s.t. \qu&	\sum w_i = 1
							& γ_{ij} = σ^2 - C_{ij} = E((Z(x_i) - Z(x_j))^2)
					$
				- 步骤:
					- 确定半方差函数$γ(xi,xj) = E((Z(x_i) - Z(x_j))^2)$
						确定半方差函数与两点间距离的函数关系.
					- 计算权值
						权值计算方程:
						$
							(\mb w_1 \\ \vdots \\ w_n \\ μ \me) = (\mb γ(x_1,x_1) & ... & γ(x_1,x_n) & 1 \\ ... & ... & ... & ... \\  γ(x_n,x_n) & ... & γ(x_n,x_n) & 1 \\ 1 & ... & 1 & 0 \me)^{-1} (\mb γ(x_1,x^*) \\ \vdots \\ γ(x_1,x^*) \\ 1 \me)
						$
					- 计算插值点结果
						$f(x) = \sum w_i(x) f(x_i)$
				- 原理:
					- $f(x) = \sum w_i(x) f(x_i)$
						- $\tilde z$ 估计值, $z$ 实际值
						求解权重系数: 使其为满足插值点处, 估计值与真实值差最小的一组系数.
					- 对空间任意一点, 都有相同期望、方差. 即:
						$
							E	(z(x,y)) = μ
							Var	(z(x,y)) = σ^2
							=> z(x,y) = μ + R(x,y)    Var(R(x,y)) = σ^2
						$
					- 约束方程 —— 无偏估计条件 $E(\tilde z - z) = 0$
						$
							=> E(\tilde z - z) = E(\sum w_i z_i - z) = μ\sum w_i - μ = 0
							=> \sum w_i = 1
						$
					- 目标函数 —— 估计误差 $var = Var(\tilde z - z)$
						$
							=> var = Var(\sum w_i z_i - z) = \sum\sum w_i w_j Cov(z_i,zj) - 2\sum w_i Cov(z_i,z) + cov(z,z)
							=> var = \sum\sum w_i w_j C_{ij} - 2\sum w_i C_i0 + cov_{00} \qu; (C_{ij} = Cov(z_i - μ, z - μ))
						$
					- 定义 半方差函数$γ_{ij} = σ^2 - C_{ij}$
						$
							=> var = 2\sum w_i γ_i0 - \sum\sum w_i w_j γ_{ij} - γ_{00}
						$
					- 凸优化问题构建完成:
						$
							\min \qu&	var = 2\sum w_i γ_i0 - \sum\sum w_i w_j γ_{ij} - γ_{00}
							s.t. \qu&	\sum w_i = 1
						$
						Lagrange函数	$L(w_i, λ) = var + λ(\sum w_i - 1)$
						Lagrange对偶	$G(λ) = \inf L(w_i, λ)$
						- $L(W,λ)$求导, 当导数为0时, 取得极值
						即. 得到权值计算方程.
					- 半方差函数$γ_{ij} = σ^2 - C_{ij} = E((Z(x_i) - Z(x_j))^2)$
						∵地理学第一定律: 空间上相近的属性相近.
						∴$γ_{ij}$与两点间距离, 存在函数关系
						将所有d和$γ$绘成散点图, 寻找最优曲线拟合d与$γ$, 得到函数关系式.
						