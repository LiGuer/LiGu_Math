* 支持向量机
	\Problem
		$
			\max_{\.w, b} \qu& \min_i d_i = \/{1}{||\.w||} \min_i |\.w^T \.x_i + b|
			s.t. \qu& y_i (\.w^T \.x_i + b) ≤ 1
		$
		找到一个超平面, 最大化样本点与超平面的最小距离, 且使得不同类别$y_i \in \{-1,1\}$的点分居超平面的两侧. 
	
	\Algorithm
		$
			\max_{\.λ} \qu& \sum λ_i - 1/2 \sum_i \sum_j λ_i λ_j y_i y_j \.x_i^T \.x_j  \tag{对偶问题}
			s.t. \qu& \sum_i λ_i^* y_i = 0
				& λ_i &≥ 0
		$
		利用Sequential Minimal Optimization算法求解$λ_i^*$, 代入解出$\.w^*, b^*$.
		
		\Proof
			优化问题简化, 可令$\min_i |\.w^T \.x_i + b| = 1$, 且$ \arg\max_{\.w, b} \/{1}{||\.w||} => \arg\min_{\.w, b} \/{||\.w||^2}{2}$
			$
				\max_{\.w, b} \qu& \/{||\.w||^2}{2}
				s.t. \qu&	y_i (\.w^T \.x_i + b) ≤ 1
			$
			Lagrange函数,
			$
				L(\.w, b,\.λ) = \/{||\.w||^2}{2} + \sum_i λ_i (1 - y_i (\.w^T \.x_i + b)) \tag{Lagrange函数}
				\/{∂ L(\.w, b,\.λ)}{∂ \.w} |_{\.w = \.w^*} = \.w^* - \sum_i λ_i x_i y_i = 0
				\/{∂ L(\.w, b,\.λ)}{∂ b} |_{b = b^*} = - \sum_i λ_i y_i = 0
			$
			$
				=>\qu& \.w^* = \sum_i λ_i y_i x_i
				& \sum_i λ_i^* y_i = 0
			$
			Lagrange对偶函数, 对偶问题,
			$
				G(\.λ) = \inf_{\.w, b} L(\.w, b, \.λ)  \tag{Lagrange对偶函数}
					= L(\.w^*, b^*, \.λ)
					= 1/2 (\sum_i λ_i y_i x_i)^T (\sum_j λ_j y_j x_j) + (-\sum_i λ_i y_i (\sum_j λ_j y_j x_j)^T x_i + \sum_i λ_i)  \tag{$\.w^*, b^*$代入}
					= \sum λ_i - 1/2 \sum_i \sum_j λ_i λ_j y_i y_j \.x_i^T \.x_j
			$
			$
				\max_{\.λ} \qu& \sum λ_i - 1/2 \sum_i \sum_j λ_i λ_j y_i y_j \.x_i^T \.x_j  \tag{对偶问题}
				s.t. \qu& \sum_i λ_i^* y_i = 0
					& λ_i &≥ 0
			$
			对偶问题是凸二次规划问题, 可利用Sequential Minimal Optimization算法求解$λ_i^*$, 代入解出$\.w^*, b^*$.
			KKT条件,
			$
				y_i (\.w^{*T} \.x_i + b^*) ≤ 1
				λ_i^* ≥ 0
				λ_i^* (1 - y_i (\.w^{*T} \.x_i + b^*)) = 0
				\.w^* = \sum_i λ_i^* x_i y_i
				\sum_i λ_i^* y_i = 0
			$

* 核方法
	* 核函数
		\def
			$κ(\.x_i, \.x_j) = \phi(\.x_i)^T \phi(\.x_j)$
		\Example
			- 线性核: $κ(\.x_i, \.x_j) = \.x_i^T \.x_j$
			- 高斯核: $κ(\.x_i, \.x_j) = e^{-1/2(\.x_i - \.x_j)^T Σ^{-1} (\.x_i - \.x_j)}$
			- 多项式核: $κ(\.x_i, \.x_j) = (\.x_i^T \.x_j)^d$
			- Laplace核: $κ(\.x_i, \.x_j) = e^{-\/{\|\.x_i-\.x_j\|}{\sigma}}$
			- Sigmoid核: $κ(\.x_i, \.x_j) = \tanh (\beta \.x_i^T \.x_j + θ)$

