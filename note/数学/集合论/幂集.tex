* 幂集
	\Define
		$\{x | x \subseteq S\}$
		对于一个集合$S$, 幂集是$S$及其所有子集的集合.

	\Examle
		- 对于一个集合$\{x, y, z\}$的幂集是
			$\{
				\{\} (\emptyset), 
				\{x\},  \{y\},  \{z\}, 
				\{x, y\},  \{y, z\},  \{z, x\}, 
				\{x, y, z\}
			\}$

	* σ域 (σ-algebra)
		\Define
			对于一个集合$Ω$, 及其幂集$PowerSet(Ω)$, 则σ域是幂集的一个子集$Σ \subseteq PowerSet(Ω)$, 且满足
			- $\emptyset \in Σ, Ω \in Σ$
				包含空集、全集 
			- if $A \in Σ$, then $complement(A) \in Σ$.
				补集封闭: 若一集合在σ域中, 则其补集也在其中.
			- if $A_1, ... , A_n \in Σ$, then $A_1 \cup ... \cup A_n \in Σ$
				可数交集封闭: 若一些集合在σ域中, 则其并集也在其中.

		\Perporty
			- 可数交集集封闭
				if $A_1, ... , A_n \in Σ$, then $A_1 \cap ... \cap A_n \in Σ$

				\Proof
					De Morgan's 定律
			- 最小σ域 $\{\emptyset, Ω\}$
				最大σ域 $PowerSet(Ω)$