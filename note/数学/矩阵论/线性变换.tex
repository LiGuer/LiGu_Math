* 线性变换
	\Define
		$T(k x + l y) = k(T x) + l(T y)$
		线性空间$V$到自身的一类映射$T$, 对于所有$x \in V$都有唯一的$y \in V$与之对应, 且满足线性条件.

		线性变换可由基构成的矩阵表示. 即, 线性变换矩阵
		$T X = X A  \qu; X = [x_1, ... , x_n]$ 

	\Property
		- 运算 
			- $(T_1 + T_2) X = X (A + B)$
			- $(k\ T_1) X = X (k\ A)$
			- $(T_1 T_2) X = X AB$
			- $T_1^{-1} X = X A^{-1}$

		- 值域 
			$Range(T)=\{T x | x \in V\}$
			线性空间中, 所有向量在线性变换后的结果的集合, 即 线性变换后的线性空间. 

			- 秩
				$rank(A) = \dim Range(A) = \dim Range(A^T)$
				变换后的空间的维数, 即 值域的维数.

		- 零空间
			$Null(T) = \{x | T x = 0\}$
			线性空间中, 所有在线性变换为零向量的原向量的集合. 

		- $\dim V = \dim Range(A) + \dim Null(A)$
			变换前线性空间维数 = 值域维数 + 零空间维数. 

		- 不变子空间
			$\forall x \in V_1, V_1 \subseteq V, T x \in V_1$

		* 特征值、特征向量
			\Define
				$T x = λ x$
				- $x$ 特征向量, 是线性变换前后方向不改变的向量;
				- $λ$ 特征值, 是特征向量在线性变换后长度变化的倍率.

			\Property
				- 特征多项式
					$\varphi(λ) = |λ I - A| = λ^n + a_1 λ^{n-1} + ... + a_{n-1} λ + a_n$

				- \Theorem Hamilton-Cayley定理
					$\varphi(A) = A^n + a_1 A^{n-1}+ ... +a_{n-1} A + a_n I = 0$
					矩阵是其特征多项式的根.

		* 广义逆
			\Define
				满足以下方程的解,
				$ 
					\{\mb
						A X A = A
						X A X = X
						(A X)^H = A X
						(X A)^H = A X
					\me\right.
				$
				列满秩时, $A^+ = (A^H A)^{-1} A^H$
				行满秩时, $A^+ = A^H (A A^H)^{-1}$

			\Property
				- $rank(A) = rank(A^H A) = rank(A A^H)$
				- 满秩分解算广义逆 $A^+ = G^H (F^H A G^H)^{-1} F^H$

		* 相似
			\Define
				$\exists \text{非奇异矩阵}P => B = P^{-1} A P$
				则$A$与$B$相似, 记作$A ~ B$.

			\Property 
				- $A ~ A$ 
				- $A ~ B <=> B ~ A$ 
				- $A ~ B, B ~ C <=> A ~ C$
				- 相似矩阵特征值、特征向量相同.
				- 相似矩阵迹相同.

		- 不同基下线性变换矩阵的转换
			$A_Y = C^{-1} A_X C \qu; Y = C X$

			\Proof
				$
					T Y &= Y A_Y		 \tag{定义} 
					T X C &= X C A_Y	 \tag{$ Y = X C $}
					X A_X C &= X C A_Y   \tag{$ T X = X A_X $}
					A_X C &= C A_Y
					A_Y &= C^{-1} A_X C
				$
				
	\Include
		* 恒等变换
			\Define
				$T x = x \qu ;(\forall x \in V)$

		* 零变换
			\Define
				$T x = 0 \qu ;(\forall x \in V)$

		* 正交变换
			\Define
				$<x, x> = <T x, T x>$
				内积空间中, 保持任意向量的长度不变的线性变换.
				正交矩阵:
					$A A^T = I$
					$A A^H = I$

			* 初等旋转变换
				\Define
					初等旋转变换矩阵:
					$T_{ij} = (\mb
						\. I \\ & cosθ|_{(i,i)}&  & \sinθ|_{(i,j)} \\ & & \. I \\ & -\sinθ|_{(j,i)} & & \cosθ|_{(j,j)} \\ & & & & \. I
					\me)$

			* 初等反射变换
				\Define
					$y = H x = (I - 2 e_2 e_2^T) x$
					\Proof
						$
							x - y &= e_2 · (e_2^T x)
							=> y &= (I-2 e_2 e_2^T) x
						$

		* 对称变换
			\Define
				$<T x, y> = <x, T y>$
				对称矩阵:
					$A^T = A$
					$A^H = A$

		* 投影变换
			\Define
				令线性空间分为不交的子空间L,M, 投影变换是将线性空间沿M到L的投影的变换.
				投影矩阵: 
					$P_{L,M} = (\mb X & 0 \me) (\mb X & Y \me)^{-1}$

			* 正交投影变换
				\Define
					设线性空间的子空间L, 将线性空间沿$L_\bot$到L的投影的变换, 称投影变换.
					正交投影矩阵:
						投影后子空间的基 $X = (x_1, ... , x_r) ,$ 则正交投影矩阵 $P_L = X(X^H X)^{-1}X^H$.

		* 斜切变换
			\Define
				斜切变换矩阵: 
					单位矩阵的第(i,j)个元素改为斜切比率 $a_{ij}$

		* 缩放变换
			\Define
				缩放变换矩阵:
				$T = (\mb Δx_1 \\ & Δx_2 \\ & & \ddots \\ & & & Δx_n \me)$