* 矩阵分解
	\Include
		* 上下三角分解
			\Problem
				将矩阵A化成上三角矩阵R与下三角矩阵L的乘积.$A = L R$

		* 上下三角对角分解
			\Problem
				将矩阵A化成上三角矩阵R, 对角矩阵D, 下三角矩阵L的乘积.$A = L D R$

		* 对称三角分解
			\Problem
				将对称正定矩阵化成对称的两个上下三角矩阵. $A = G G^T$

			\Algorithm
				- 先上下三角分解 $A = L D U = L D L^T \tag{因为对称正定矩阵}$
				- 
					$
						A &= L (\sqrt{D})^2 L^T
							&= (L \sqrt{D}) (\sqrt{D} L^T)
							&= (L \sqrt{D}) (L \sqrt{D})^T
							&= G G^T
					$

		* 正交三角分解
			\Problem
				将非奇异矩阵A化成正交矩阵Q与非奇异上三角矩阵R的乘积. $A = Q R$

			\Algorithm
				- Schmidt正交化方法
					- $A = (a_1, ..., a_n)$
					- Schmidt正交化 $b_i = a_i - \sum_{k=1}^{i-1} \/{<a_i,b_j>}{<b_j,b_j>}b_j$
					- 
						$
							Q &= ( \/{b_1}{|b_1|}, ... , \/{b_n}{|b_n|} )
							R &= (\mb |b_1|\\ & \ddots\\ && |b_n| \me) (\mb 1 & k_{21} & ... & k_{n1} \\ & 1 & ... & k_{n2} \\& & \ddots & \vdots \\& & & 1 \me) \qu; k_{ij} = \/{<a_i,b_j>}{<b_j,b_j>}
							A &= Q R
						$

				- 初等旋转变换方法
					- 对第1列, 初等旋转变换使其变为 $T_i a_1 = (b_{11}, 0,...,0)$
					- $T_i = \prod_{i=0}^{n-1} T_{i(n-1-j)}$
					- 重复上面步骤, 直至将 $A_i$ 化为上三角矩阵
					- 
						$
							R &= A_{n-1}
							Q &= (\prod_{i=0}^{n-1} T_{n-1-i} )^T
							A &= Q R
						$

				- 初等反射变换方法
					- 对第1列, 初等旋转变换使其变为 $H_i a_1 = (b_{11}, 0,...,0)$
						$
							u_i &= \/{b_i - |b_i|}{| b_i - |b_i| |}
							H_i &= I - 2 u u^T
							A_{i+1} &= H_i A_i
						$
					- 重复上面步骤, 直至将 $A_i$ 化为上三角矩阵
					- 
						$
							R &= A_{n-1}
							Q &= (\prod_{i=0}^{n-1} H_{n-1-i} )^T
							A &= Q R
						$

		* 满秩分解
			\Problem
				将矩阵A化成F G的乘积. $A = F G$

				\Proof
					$A =P^{-1} B = (\mb F & S \me) (\mb G\\ 0 \me) = F G$
					
			\Algorithm
				初等行变换, 取A左侧rank(A)列作为F, 则$A = F G$
					$A \to (\mb G \\ 0 \me)$

		* 奇异值分解
			\Theorem
				$\exists \text{Unitary矩阵} U, V => U^H A V = (\mb Σ & 0 \\ 0 & 0 \me)$

			\Problem
				将矩阵A化成两个Unitary矩阵$U, V$, 和一个非零奇异值组成的矩阵$Σ$的乘积. 
					$A = U (\mb Σ & 0 \\ 0 & 0 \me) V^T$

			\Algorithm
				- $A^T A$ 计算特征值 $λ$, 特征向量$x$
				- 
					$V = ( \/{x_1}{|x_1|}, ... ,\/{x_n}{|x_n|} ), \qu Σ = diag(\sqrt{λ_1}, ... ,\sqrt{λ_n})$
				- $U_1 = A V Σ^{-1}$, 计算正交矩阵$U$
				- 
					$ A = U (\mb Σ & 0 \\ 0 & 0 \me) V^T $

			\Property
				$
					Range(A) &= Span(u_1, ..., u_r)
					Null (A) &= Span(v_{r+1}, ... , v_n)
					Range(A^T) &= Span(v_1, ..., v_r)
					Null (A^T) &= Span(u_{r+1}, ... , u_m)
					A &= \sum_{i=1}^{r} σ_i u_i v_i^H
				$
				
				\Proof
					$
						A &= (\mb U_{1:r} & U_{r+1:m} \me) (\mb Σ & 0 \\ 0 & 0 \me) (\mb V_{1:r}^H \\ V_{r+1:n}^H \me)  \tag{定义式变形}
							&= U_{1:r} Σ V_{1:r}^H
					$
					$
						Range(A) &= \{y\ |\ A x = y\}
							&= \{y\ |\ U_{1:r} (Σ V_{1:r}^H x) = y\}  \tag{代入}
							&\subseteq Range(U_{1:r})
						Range(U_{1:r}) &= \{y\ |\ U_{1:r} x = y\}
							&= \{y\ |\ A (V_{1:r} Σ^{-1} x) = y\}  \tag{$U_{1:r} &&= A V_{1:r} Σ^{-1}$}
							&\subseteq Range(A)
					$
					$
						=> Range(A) = Range(U_{1:r}) = Span(u_1, ..., u_r)
					$