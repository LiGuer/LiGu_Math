* 数论
	* 整数环
		\Define
			$(\mathbb Z, +, ·)$

	* 带余除法
		\Define
			$
				a &= q · b + r
				a,b,q,r &\in \mathbb Z, 0 ≤ r < b
				q &= a / b  \tag{商, $ / $除}
				r &= a % b  \tag{余数, $ % $取余}
			$

		\Property
			- 右分配律
				$
					(a + b) % c &= (a % c + b % c) % c
					(a - b) % c &= (a % c - b % c + c) % c
					(a · b) % c &= ((a % c) · (b % c)) % c
					(a / b) % c &= (a · b^{-1}) % c = ((a % c) · (b^{-1} % c)) % c  \tag{$b^{-1}$:$b$的逆元}
				$
				\Note
					计算减法时,通常需要$+ c$ ,防止变成负数.

		- 同余方程组
			\Problem
				$
					\{\mb
						x % m_1 = a_1
						\vdots
						x % m_n = a_n
					\me\right.
				$

			\Answer
				$x = k \prod_{i=1}^n m_i + \sum_{i=1}^n a_i (\prod_{j=1, j≠i}^n m_j) (\prod_{j=1, j≠i}^n m_j)^{-1}$

		- 高次同余方程
			\Problem
				$(x^p) % m = a$

	* 质数
		\Define
			只能被1和自身整除的数.

		- 算数基本定理
			$n = \prod_i p_i^{α_i} \qu; n \in \mathbb Z, n > 1$
			任何一个大于1的整数,都可以唯一地表示成素数乘积的形式

			* 分解质因数
				\Problem

				\Algorithm
					- Pollard Rho 算法

* 积性函数
	\Define
		映射 $f: \mathbb Z \to R$, 且满足
		$
			f(a · b) = f(a) f(b) \qu when\ a, b \in \mathbb Z, gcd(a, b) = 1
		$

	\Property
		- $f(1) = 1$

	* Eular函数
		\Define
			$\phi(n) = number({i\ |\ i \in 1:n, gcd(i, n) = 1})$
			小于n的正整数中, 和n互质的数的个数.

		\Property
			- Eular函数是积性函数
			- 
				$
					n &= \prod_i p_i^k_i
					\phi(n) &= n \prod_{p|n} (1 - 1/p)
				$

- Fermat's小定理
	$a^{p-1} % p = 1$

* 最大公因数
	\Property
		$
			gcd(a, b) &= gcd(b, a)
			gcd(a, b) &= gcd(a - b, b) \qu;(a ≥ b)
			gcd(a, b) &= gcd(a % b, b)
			gcd(k a, k b) &= gcd(k a, b)
		$

	- 求最大公因数
		\Problem
		
		\Algorithm
			- 辗转相除法
				$
					gcd(a, b) &= gcd(b, a % b)
					gcd(a, 0) &= a
				$

* 最小公倍数

	- 求最小公倍数
		\Problem
			
		\Algorithm
			$lcm(a, b) = \/{a · b}{gcd(a, b)}$

			\Note
				先除后乘$a / gcd(a,b) · b$,减少中间过程数的位数.

* 逆元
	\Define
		$(a · c) % b = 1$
		则$c$是$a$在$mod\ b$下的逆元$a^{-1}$

	- 求逆元
		\Problem
			$(a · a^{-1}) % b = 1$
			知$a, b$求$a^{-1}$.

		\Proerty
			- 问题等价于\Problem{解二元一次整数方程}
				$
					=> a × a^{-1} + (-y) × b = 1
				$
			- 存在性: $a, b$互质 $ <=> $ 逆元存在.
				\Proof 解二元一次整数方程的性质可得.

		\Algorithm
			- Fermat's小定理
				$	
					a^{b-1} % b = 1  \tag{Fermat's小定理}
					(a × a^{b-1}) % b = 1
				$

* 解二元一次整数方程
	\Problem
		$
			a x + b y = c
			a, b, c, x, y \in mathbb Z
		$
		知$a, b, c$, 求$x, y$.
	
	\Property
		- 方程有解 $ <=> gcd(a, b) % c = 0 $
		- 非零正整数$a, b$, 总存在$x, y$满足 $a x + b y = gcd(a, b)$. 
		- $a, b$互质$ => a x + b y = 1$

	\Algorithm
		- 答案
			$
				x &= x_0 + b/d t
				y &= y_0 - a/d t
			$
			- $x_0, y_0$ 是一组特解
			- $d = gcd(a, b)$

		- 扩展Euclid算法
			$
				f(x, y) &= a x + b y
				f(1, 0) &= 1 · a + 0 · b = a  \tag{初始易知值}
			$

* 幂次模
	\Problem
		$b = (a^k) % m$

	\Algorithm
		- 逐次平方法
			- 步骤
				- 将 k 二进制展开
					$k = \sum_{i=0}^r u_i·2^i$

					\Note
						计算机里, k内存天然是二进制

				- 逐次平方制作模$m$的$a$幂次表, $i\in[0,r]$
					$
						a^(2^0) &= a = A_0 % m
						a^(2^i) &= (a^2^(i-1))^2 = A^2(i-1) = A_i % m
					$

				- 乘积
					$\prod_{i=0}^r A_i^{u_i} % m$
			
			\proof
				$a^k = a^{\sum_{i=0}^r u_i·2^i}$

* Möbius函数
	\Property
		- Möbius反演