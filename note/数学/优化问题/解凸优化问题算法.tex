* 解凸优化问题算法
	* 解线性规划
		* 单纯形法
			\Note
				因为线性规划, 可行解集是一个高维单纯形(最简多面体), 最优解在多面体的顶点或边界上, 而不在多面体内部. 所以沿着单纯形的每一个顶点寻找最优解的一种方法. 
			- 大M法

	* 解无约束凸优化问题
		- 
			$\min_{\.x}\ f(\.x)$
			其中$f(\.x)$是可微凸函数

		- 最优性条件
			$∇ f(\.x^*) = 0$
			\Note
				- KKT条件同样可得. $∇ L = ∇ f(\.x^*) = 0$

		* 下降法
			* 梯度下降法
				$x_{k+1} = x_k - λ ∇ f(x_k)$

				* 最速下降法
					$x_{k+1} = x_k + \arg\min \{ (∇ f(x_k))^T v\ |\ ||v|| = 1 \}$

				* 牛顿迭代法
					$x_{k+1} = x_k - \frac{1}{∇^2 f(x)} ∇ f(x_k)$

				* 拟牛顿法

			* 坐标下降法
				* 块坐标下降

	* 解等式约束凸优化问题
		- 方法: 先消除等式约束, 简化为无约束凸优化问题.
		- 方法: 先求其对偶问题

	* 解不等式+等式约束凸优化问题
		- 
			$
				\min \qu& f_0(\.x)
				s.t. \qu& f_i(\.x)
					& \.A \.x = \.b
			$
			目标函数、约束函数都是二次可微凸函数. 
		\Property
			- KKT条件
				$
					f_i(\.x^*) &≤ 0
					\.A \.x &= \.b
					\.λ^* &⪰ 0
					λ_i^* f_i(\.x^*) &= 0
					∇ L = ∇ f_0(\.x^*) + \sum_{i=1}^m λ_i^* ∇ f_i(\.x^*) + \.A^T \.v^* &= 0
				$

		* 内点法
			* 障碍函数法
				\Algorithm
					利用障碍函数$-1/t \log (-f_i(\.x))$, 将不等式约束转换到目标函数中去, 从而转化为等式约束凸优化问题, 求解其最优解$\.x^*(t)$. 同时, $\.x^*(t)$是与原问题最优解偏差在$m/t$以内的次优解, $t \to \infty, \.x^*(t) \to \.x^*$.
						$
							\min \qu& f_0(\.x) - 1/t \sum_i \log (-f_i(\.x))
							s.t. \qu& \.A \.x = \.b
						$

					- 序列无约束极小化方法: 顺序求解一系列障碍函数新问题$\.x^*(t)$(无约束极小化问题), 每次用所获得的最新点组为求解下一个问题的初始点, 并逐渐增加$t = t + Δt$直到$t ≥ m/ε$.

					- 步骤 (序列无约束极小化方法)
						- 初始化
							严格可行点$\.x$, $t_{(0)}>0$, $Δt$, 误差阈值$ε>0$
						- 循环 ($m/t < ε$时,退出循环)
							从$\.x$开始, 求解障碍函数新问题的解 $\.x^*(t)$. 改进 $\.x = \.x^*(t)$, $t = t + Δt$.

				\Note
					- 将不等式约束转换到目标函数中去. $I_{-}(\.x)$是指示函数, (1) 当$\.x ≤ 0$ 满足原不等式约束, $I_{-}(\.x) = 0$对原目标函数不产生影响; (2) 当$\.x > 0$ 违背原不等式约束, $f_0(\.x) + \sum_i I_{-}(f_i(\.x)) = \infty$, 所以该点无法作为最优点. 综上所述, 指示函数的新优化问题与原问题等价.
						$
							=> \min \qu& f_0(\.x) + \sum_i I_{-}(f_i(\.x)) 
								s.t. \qu& \.A \.x = \.b
						$
						$
							I_{-}(\.x) = \{\mb
								0  \qu, \.x ≤ 0
								\infty  \qu, \.x > 0
							\me\right.  \tag{指示函数}
						$
					- 但是, 指示函数不可微, 因此想办法用可微的函数去拟合指示函数, 来达到相同效果. 对于每一个设置的$t$, 都有一个问题的最优解$x^*(t)$与之对应. $t$越大, 对$I_{-}(x)$拟合的越好, 新的优化问题的解就越接近于原问题.
						$
							I_{-}(x) \approx -1/t \log (-x)  \tag{$t>0$}
						$
						$
							=> \min \qu& f_0(\.x) - 1/t \sum_i \log (-f_i(\.x))
								s.t. \qu& \.A \.x = \.b
						$

					- KKT条件 (障碍函数新问题)
						$
							\.A \.x^*(t) &= \.b
							∇ L &= ∇ f_0(\.x^*(t)) + 1/t ∇(- \sum_i \log (-f_i(\.x))) + \.A^T \.v^*(t)
								&= ∇ f_0(\.x^*(t)) + 1/t \sum_{i=1}^m \/{∇ f_i(\.x^*(t))}{-f_i(\.x^*(t))} + \.A^T \.v^*(t)
								&= 0
						$
						原问题KKT条件, 与障碍函数新问题KKT条件, 对比如下. 令$λ_i^*(t) = \/{1}{-t f_i(\.x^*(t))}$, 则二者表达式一样.
						$
							∇ f_0(\.x^*(t)) + 1/t \sum_{i=1}^m \/{∇ f_i(\.x^*(t))}{-f_i(\.x^*(t))} + \.A^T \.v^*(t) &= 0
							∇ f_0(\.x^*) + \sum_{i=1}^m λ_i^* ∇ f_i(\.x^*) + \.A^T \.v^* &= 0
						$

					- $\.x^*(t)$是与原问题最优解偏差在$m/t$以内的次优解, $t \to \infty, \.x^*(t) \to \.x^*$
						\Proof
							只要证 $0 ≤ f_0(\.x^*(t)) - f_0(\.x^*)) ≤ m/t$.
							上文原问题与障碍函数新问题KKT条件对比可得, $\.x^*(t)$使得$L(\.x, \.λ^*(t), \.v^*(t))$在其位置达到极小值, 
							$∇ L(\.x^*(t), \.λ^*(t), \.v^*(t)) = 0$
							$
								=> \qu g(\.λ^*(t), \.v^*(t)) &= \inf_{x \in D} L(\.x, \.λ^*(t), \.v^*(t))
									&= L(\.x^*(t), \.λ^*(t), \.v^*(t))
									&= f_0(\.x^*(t)) + \sum_{i=1}^m λ^*_i(t) f_i(\.x^*(t)) + \.v^*(t)^T (\.A^T \.x^*(t) - \.b)
									&= f_0(\.x^*(t)) + \sum_{i=1}^m \/{1}{-t \.x^*(t)} f_i(\.x^*(t))  \tag{$\.A^T \.x^*(t) - \.b = 0$}
									&= f_0(\.x^*(t)) - m/t
							$
							又$\because$ 强对偶性有, $g(\.λ^*, \.v^*) = f_0(\.x^*)$
							$
								\therefore \qu& g(\.λ^*(t), \.v^*(t)) ≤ g(\.λ^*, \.v^*) = f_0(\.x^*)
								=> \qu&  f_0(\.x^*(t)) - m/t ≤ f_0(\.x^*)  \tag{代入}
								=> \qu&  f_0(\.x^*(t)) - f_0(\.x^*)) ≤ m/t
							$
							又$\because$ 原问题目标函数是$\min f_0(\.x)$
							$
								=> \qu&  f_0(\.x^*(t)) ≥ f_0(\.x^*)
								=> \qu&  0 ≤ f_0(\.x^*(t)) - f_0(\.x^*) ≤ m/t
							$

			* 原始对偶法
