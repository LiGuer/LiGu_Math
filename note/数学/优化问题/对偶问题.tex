* 对偶问题
	\def
		$
			\max_{\.λ, \.ν} \qu& g(\.λ, \.ν)
			s.t. \qu& \.λ ⪰ 0
		$
		* Lagrange 函数
			\def
				$L(\.x, \.λ, \.ν) = f_0(\.x) + \sum_i λ_i f_i(\.x) + \sum_i ν_i h_i(\.x)$
				- $f_0, f_i, h_i$是标准形式优化问题的目标函数、约束函数.

		* Lagrange 对偶函数
			\def
				$
					g(\.λ, \.ν) = \inf_{\.x \in D}\ L(\.x, \.λ, \.ν)
						= \inf_{\.x \in D}\ f_0(\.x) + \sum_i λ_i f_i(\.x) + \sum_i ν_i h_i(\.x)
				$
				- $L: \bb R^n × \bb R^m × \bb R^p \to \bb R,\qu dom\ L: D × \bb R^m × \bb R^p$
			\Property
				- 即使原问题非凸, 对偶函数仍然是凹函数.
				- $g(\.λ, \.ν) ≤ p^*$
					对偶函数构成原问题的下界. 
					对偶函数是一族关于$(\.λ, \.ν)$的仿射函数的逐点下确界.
					\Proof
						设$\tilde{\.x}$是原问题的一个可行点,
						$
							=> λ_i f_i(\tilde{\.x}) ≤ 0  \tag{$\.λ ⪰ 0, f(\tilde{\.x}) ≤ 0$}
								ν_i h_i(\tilde{\.x}) = 0  \tag{$h(\tilde{\.x}) = 0$}
							=> \sum_i λ_i f_i(\tilde{\.x}) + \sum_i ν_i h_i(\tilde{\.x}) ≤ 0
						$
						$
							=> L(\tilde{\.x}, \.λ, \.ν) = f_0(\tilde{\.x}) + \sum_i λ_i f_i(\tilde{\.x}) + \sum_i ν_i h_i(\tilde{\.x}) ≤ f_0(\tilde{\.x})
							=> g(\.λ, \.ν) ≤ p^*
						$
	\Property
		- 对偶性
			- 强对偶性: $p^* ≥ d^*$ 一定存在
			- 弱对偶性: $p^* = d^*$
		* Slate 准则
		* KKT最优性条件
			\def
				$
					f_i(\.x^*) &≤ 0  \qu, i = 1,...,m  \tag{满足原问题约束}
					h_i(\.x^*) &= 0  \qu, i = 1,...,p
					\.λ^* &⪰ 0  \tag{满足对偶问题约束}
					λ_i^* f_i(\.x^*) &= 0 \qu, i = 1,...,m  \tag{$L(\.x^*, \.λ^*, \.ν^*) = f_0(\.x^*)$对偶间隙为零}
					∇L(\.x^*, \.λ^*, \.ν^*) &= ∇ f_0(\.x^*) + \sum_i λ_i^* ∇ f_i(\.x^*) + \sum_i ν_i^* ∇ h_i(\.x^*) = 0  \tag{$L(\.x, \.λ^*, \.ν^*)$ 在$\.x^*$取极值}
				$
				其中, 目标函数$f_0$和约束函数$f_i, h_i$可微.
				
				- 若原问题是非凸, 若$\.x^*$和$(\.λ^*, \.ν^*)$是原问题、对偶问题最优解, 则$\.x^*, \.λ^*, \.ν^*$满足KKT条件.
				- 若原问题是凸, $\.x^*, \.λ^*, \.ν^*$满足KKT条件 $ <=> $ $\.x^*$和$(\.λ^*, \.ν^*)$是原问题、对偶问题最优解.

				\Proof
					- (1)(2) 式, 说明$\.x^*$是原问题的可行解.
					- (3) 式, 说明$\.x^*$是对偶问题的可行解. 又$\because$ 原问题是凸问题, $\therefore$ $L()$函数是$\.x$的凸函数.
					- (5) 式, 说明$L(\.x, \.λ^*, \.ν^*)$ 在$\.x^*$处取得极值. 又$\because$ $L()$函数是$\.x$的凸函数, $\therefore$ $L()$ 在$\.x^*$取得极小值. 
						$
							=> g(\.λ^*, \.ν^*) &= \inf_{\.x} L(\.x, \.λ^*, \.ν^*)
								&= L(\.x^*, \.λ^*, \.ν^*)  \tag{$L()$在$\.x^*$取极小}
						$
					- (4) 式, 说明$\.x^*$处, $L(\.x^*, \.λ^*, \.ν^*) = f_0(\.x^*)$
						$ => g(\.λ^*, \.ν^*) = f_0(\.x^*)  \tag{代入}$
						又$\because$ 弱对偶性有, 
						$\max_{\.λ, \.ν} g(\.λ, \.ν) ≤ \min_{\.x} f_0(\.x)$
						.$\therefore$ 强对偶性成立, 且在$\.x = \.x^*$和$(\.λ^*, \.ν^*)$处取得原问题和对偶问题的最优值, 对偶间隙为零.
				
	\Note Lagrange 对偶问题理解 (Boyd,凸优化,233页)
		\Fig "Lagrange对偶问题理解.jpg"

		- 优化问题
			目标函数不一定是凸函数
			$
				\min \qu& f_0(\.x)  \tag{目标函数}
				s.t. \qu& f_i(\.x) ≤ 0  \qu i = 1,...,m  \tag{不等式约束}
					& h_i(\.x) = 0  \qu i = 1,...,p  \tag{等式约束}
			$

		- 可行集
			$
				G = \{(f_1(x), ... , f_2(x), h_1(x), ... , h_2(x), f_0(x)) \in \bb R^m × \bb R^n × \bb R \ |\  x \in D\}
					= \{(\.u, \.v, t) \ |\  u_i = f_i(x), v_i = h_i(x), t = f_0(x), x \in D\}
			$
			图1. 以只有一个不等式约束为例, 可行集$G$区域如图所示.

		- 原问题
			$
				p^* = \inf\{t \ |\  (\.u, \.v, t) \in G, \.u ⪯ \.0, \.v = \.0\}
			$
			图2. 因为$\.u ⪯ \.0$, 所以原问题$f_0(\.x)$ 最优值如图中$p^*$所示.

		- Lagrange 函数
			$
				L(\.x, \.λ, \.ν) = (\.λ, \.ν, 1)^T (\.u, \.v, t)  \tag{$(\.u, \.v, t) \in G, \.λ ⪰ 0$}
					= \sum_i λ_i u_i + \sum_i ν_i v_i + t
			$
			图3. 对于可行集$G$中任意一点$(u, t) \in G$, Lagrange 函数的值是经过该点$(u, t)$以斜率$k=-λ$的直线, 与纵坐标$t$的交点值. 同时, 因为$λ ⪰ 0$, 所以直线只能斜向下or水平.

		- 对偶函数
			$
				g(\.λ, \.ν) = \inf_{\.x}\ L(\.x, \.λ, \.ν)
					= \inf_{\.x}((\.λ, \.ν, 1)^T (\.u, \.v, t))
			$
			图4. 对偶函数$g(λ_0)$是在直线斜率$λ = λ_0$固定的情况下, Lagrange 函数的最小值, 即点$(u, t)$在可行集$G$内时, 直线与纵坐标$t$ 最低的交点的值. 同时, 使得直线的可行集$G$下半部分的斜向下的切线.

		- 对偶问题
			$
				d^* = \max_{\.λ, \.ν} \qu g(\.λ, \.ν)
			$
			图5. 对偶问题是"最大的最小值", 对偶函数的最大值, $ \max_{\.λ, \.ν} \inf_{\.x}\ L(\.x, \.λ, \.ν)$. 即, 找到可行集$G$下半部分的一条斜向下的切线, 使其与纵坐标的交点的值最大. 如图中$d^*$所示.

		- 强对偶性判定
			弱对偶性: 一定有$p^* ≥ d^*$
			强对偶性: $p^* = d^*$
			图6. 图中显示了该问题的对偶性强弱, 因为最优对偶间隙$p^* - d^* > 0$, 故该问题不满足强对偶性.

