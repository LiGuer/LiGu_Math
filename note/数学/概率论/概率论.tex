* 概率论
	* 概率空间
		\def
			概率空间是一个三元组 $<Ω, \mathcal F, \P>$. 
			.$Ω$ 样本空间; $\P$, 概率; $\mathcal F$ 被选择的样本集合的集合, 且满足
				- 包含空集、样本空间全集 $\emptyset, Ω \in \mathcal F$
				- 取补封闭, 如果一个事件A在其中, 那么补集也需要在其中.  
					$A \in \mathcal F => A^C \in \mathcal F$
				- 可列并封闭 
					$A_1, A_2, ... \in \mathcal F => \bigcup_{i=1}^∞ A_i \in \mathcal F$
			* 概率
				\def
					$\P: \mathcal F \to [0, 1]$
					概率是一种集合函数, 是一种对集合的测度, 且满足Kolmogorov公理.
					* Kolmogorov公理
						- 非负性 $\P(A) \in [0, 1] \qu ; \forall A \in F$
						- 规范性 $\P(Ω) = 1$
						- 可列可加性 $\P (\bigcup_i A_i)=\sum_i \P(A_i)$
				* 联合概率, 条件概率
					* 联合概率
						\def 
							.$\P(A B)$, $A B$一起发生的概率.
					* 条件概率
						\def
							.$\P(B | A)$, $A$发生条件下, $B$发生的概率.
					\Property
						* 独立性 $ <=> \P(A B) = \P(A) \P(B)$. 即, $A, B$的发生互不影响.
						- 联合概率 - 条件概率关系
							$
								\P(B | A) = \/{\P(A B)}{\P(A)}
								\P(A B) = \P(B | A) \P(A) = \P(A | B) \P(B)
							$
						* \Theorem{全概率公式}
							$\P(A) = \sum_i \P(A|B_i) \P(B_i) \qu; \sum_i A_i = Ω$
						* \Theorem{Bayes公式}
							$
								\P(A | B) = \/{\P(B | A) \P(A)}{\P(B)}
								\P(A_i | B) = \/{\P(B | A_i) \P(A_i)}{\sum_j \P(B|A_j) \P(A_j)}; \sum_j A_j = Ω
							$
				\Theorem{大数定律}
					* 弱大数定理
						$\lim_{n \to ∞} \P(|1/N \sum_{k=1}^n X_k-μ|<ε)=1$
					* Bernoulli大数定理
						$\lim_{n \to ∞} \P(|\/{f_A}{n}-p|<ε) = 1$
				\Theorem{中心极限定律}
					$\lim_{n \to∞} F_n(x) =\lim_{n \to∞} \P(\/{\sum_{k=1}^n X_k - n μ}{\sqrt{n} σ} ≤ x)=\int_{-∞}^x \/{1}{\sqrt{2 π}} e^{-t^2 / 2} \d t=\Phi(x)$
	* 随机变量
		\def	
			$X:Ω \to R$
			随机变量是一种函数.
			\Note
				事件$\{ζ | X(ζ) ≤ x\} \text{, 简写} \{X ≤ x\}$.
			* 多维随机变量
		\Property
			* 累计分布函数
				\def
					$
						F_X(x) = \P(X ≤ x)
						F_{\.X}(x) = \P(X_1 ≤ x_1, ..., X_n ≤ x_n) = \P(\.X ≤ \.x)
					$
			* 概率密度函数
				\def
					$
						f_X(x) = \/{\d F_X(x)}{\d x}
						f_{\.X}(\.x) = \/{∂^n F_{\.X}(x)}{∂ x_1 ... ∂ x_n}
					$
				* 边缘概率密度函数
					\def
						$
							f_{X_i}(x_i) = \int_{-∞}^∞ ... \int_{-∞}^∞ f_{\.X}(\.x) \d x_1 ... \d x_j ... \d x_n |_{j ≠ i}
						$
			* 矩
				\def
					- K阶矩
						$
							E(X^k) = \int_{-∞}^∞ x^k f_X(x) \d x  \tag{连续式}
								= \sum_i x_i^k \P_X(x_i)  \tag{离散式}
						$
					- K阶中心矩
						$
							E((X-μ)^k) = \int_{-∞}^∞ (x-μ)^k f_X(x) \d x  \tag{连续式}
								= \sum_i (x_i-μ)^k \P_X(x_i)  \tag{离散式}
						$
				* 联合矩
					\def
						$
							E(X^i Y^j) tag{ij阶联合矩}
							E((X-\bar X)^i (Y-\bar Y)^j) tag{ij阶联合中心矩}
						$
					\Example
						* 协相关
							$
								E(XY)  &\tag{协相关}
								E(\.X \.X^T) = [\mb E(X_i X_j) & ... \\ \vdots & \ddots \me] \text{, 简写} R_{\.X\.X} \tag{自协相关矩阵}
							$
						* 协方差
							$
								E((X-\bar X) (Y-\bar Y)) &\text{, 简写} Cov(X,Y)  \tag{协方差}
								E((\.X - \bar{\.X}) (X - \bar{\.X})^T) = [\mb E((X_i - \bar X_i) (X_j - \bar X_j)) & ... \\ \vdots & \ddots \me] \text{, 简写} \.Σ_{\.X\.X}  \tag{自协方差矩阵}
							$
							* 协相关系数
								$\rho = \/{Cov(X,Y)}{Cov(X,X) Cov(Y,Y)}$
							\Property
								- $\.Σ_{\.X\.X} = \.R_{\.X\.X} - \bar{\.X} \bar{\.X}^T$
								- 自协方差矩阵$Σ_{\.X\.X}$是半正定矩阵.
								- $\.Y = \.A \.X => \.Σ_{\.Y\.Y} = \.A \.Σ_{\.X\.X} \.A^T$
				* 期望
					\def
						$	
							E(X) = \sum x_i \P_X(x_i)  \tag{离散式}
								= \int_{-∞}^∞ x f_X(x) \d x  \tag{连续式}
							E(\.X) = [\mb \bar X_i \\ \vdots \me] 
						$
						- 条件期望
					\Property
						$
							E(Y) = \int_{-∞}^∞ g(x) f_X(x) \d x
							Y = g(X)  \tag{$g$是可测函数}
						$
		\Example
			* 离散概率分布
				* 0-1分布
					\def
						$\P(X = k) = p^k (1 - p)^k \qu; k \in \{0,1\}$
					\Property
						$
							\E(x) = p
							D(x) = p (1 - p)
						$
				* 二项分布
					\def
						$\P(X = k) = C^k_n p^k (1-p)^{n-k}$
					\Property
						$
							\E(x) = n p
							D(x) = n p(1 - p)
						$
				* 几何分布
					\def
						$\P(X = k) = p (1-p)^{n-1}$
						描述连续独立重复实验中, 首次成功所进行的实验次数.
					\Property
						$
							\E(x) = 1/p
							D(x) = \/{1 - p}{p^2}
						$
				* 超几何分布
					\def
						$\P(X = k) = \/{C_M^k C_{N_M}^{n-k}}{C_N^n}$
					\Property
						$
							\E(x) = \/{n M}{N}
							D(x) = \/{n M}{N} (1 - M/N)\/{N - M}{N - 1}
						$
				* Poisson分布
					\def
						$\P(X = k) = \/{λ^k}{k!} e^{-λ}$
					\Property
						$
							\E(x) = λ
							D(x) = λ
						$
					\Theorem{Poisson定理}
						$\lim_{n \to ∞, p \to 0, λ = n p} \/{C_M^k C_{N_M}^{n-k}}{C_N^n} = \/{λ^k}{k!} e^{-λ}$
			* 连续概率分布
				* 均匀分布
					\def
						$
							f(x) = \{\mb \/{1}{b-a} &\qu a < x < b \\ 0 &\qu other \me\right.
							F(x) = \{\mb 0 &\qu x < a \\ \/{1}{b-a} &\qu a ≤ x < b \\ 1 &\qu b ≤ x \me\right.
						$
					\Property
						$
							\E(x) = \/{a + b}{2}
							D(x) = \/{(b - a)^2}{12}
						$
				* Normal分布
					\def
						$
							f_X(x) = \/{1}{\sqrt{2 π} \sigma} e^{-\/{(x - μ)^2}{2 \sigma^2}} \qu; x \in (-∞, +∞)
							f_{\.X}(\.x) = \/{1}{(2 π)^{n/2} |\.Σ|^{1/2}} e^{-\/{(\.x - \.μ)\.Σ^{-1}(\.x - \.μ)^T}{2}} \tag{多元Normal分布}
						$
						* 标准Normal分布
							$
								f_X(x) = \/{1}{\sqrt{2 π}} e^{-\/{x^2}{2}} \qu; x \in (-∞, +∞)
							$
							$μ = 0, \sigma^2 = 1$
					\Property
						$
							\E(x) = μ
							D(x) = \sigma^2
							D(\.x) = \.Σ
						$
						\Proof
							$
								D(x) = \/{1}{(2 π)^{D/2}} \/{1}{|\.Σ|^{1/2}} \sum_{i=1}^D \sum_{j=1}^D \.u_i \.u_j^T \int e^{-\sum_{k=1}^D \/{\.y_k^2}{2 λ_k}} y_i y_j \d \.y 
									= \/{1}{(2 π)^{D/2}} \/{1}{|\.Σ|^{1/2}} \sum_{i=1}^D \.u_i \.u_i^T \int e^{-\sum_{k=1}^D \/{y_k^2}{2 λ_k}} y_i^2 \d \.y  \tag{$i ≠ j, \.u_i \.u_j^T=0, \.u_i \.u_j$ 正交}
									= \/{1}{(2 π)^{D/2}} \/{1}{|\.Σ|^{1/2}} \sum_{i=1}^D \.u_i \.u_i^T \int \prod_{k=1}^D e^{-\/{y_k^2}{2 λ_k}} y_i^2 \d \.y 
									= \/{1}{(2 π)^{D/2}} \/{1}{|\.Σ|^{1/2}} \sum_{i=1}^D \.u_i \.u_i^T(\int_{-∞}^{+∞} e^{-\/{y_i^2}{2 λ_i}} y_i^2 \d y_i \times \prod_{k=1, k ≠ i}^D \int_{-∞}^{+∞} e^{-\/{y_X^2}{2 λ_k}} \d y_k)  \tag{积分乘法结合律}
									= \/{1}{(2 π)^{D/2}} \/{1}{|\.Σ|^{1/2}} \sum_{i=1}^D \.u_i \.u_i^T((2 π λ_i)^{1 / 2} · λ_i \times \prod_{k=1, k ≠ i}^D(2 π λ_k)^{1 / 2})  \tag{见下面推导}
									= \/{1}{(2 π)^{D/2}} \/{1}{|\.Σ|^{1/2}}  · ((2 π)^{D/2} \prod_{k=1}^D λ_k) · (\sum_{i=1}^D \.u_i \.u_i^T λ_i) 
									= \sum_{i=1}^D \.u_i \.u_i^T λ_i  \tag{$\prod_{k=1}^D λ_k=|\.Σ|^{1/2}$}
									= \.Σ
							$
							当$k = i$时,
							$
								\int_{-∞}^{+∞} e^{-\/{y_λ^2}{2 λ_i}} y_i^2 \d y_i =(λ_i \sqrt{2 λ_i}) · \int_{-∞}^{+∞}(\/{y_i^2}{2 λ_i})^{1/2} e^{-\/{y_λ^2}{2 λ_i}} \d \/{y_i^2}{2 λ_i} 
								= (λ_i \sqrt{2 λ_i}) · 2 Γ(3/2)  \tag{$Γ(z) = \int_0^{+∞} x^{z-1} e^{-x} \d x$}
								= \sqrt{2 π λ_i} · λ_i  \tag{$Γ(3/2) = \/{\sqrt{π}}{2}$}
							$
							当$k ≠ i$时,
							$
								\int_{-∞}^{+∞} e^{-\/{y_λ^2}{2 λ_k}} \d y_k =(\sqrt{\/{λ_k}{2}}) · \int_{-∞}^{+∞}(\/{y_k^2}{2 λ_k})^{-1/2} e^{-\/{y_λ^2}{2 λ_k}} \d \/{y_k^2}{2 λ_k} 
								= (\sqrt{\/{λ_k}{2}}) · 2 Γ(1/2)  \tag{$Γ(z) = \int_0^{+∞} x^{z-1} e^{-x} \d x$}
								= \sqrt{2 π λ_k}  \tag{$Γ(1/2) = \sqrt{π}$}
							$
				* Rayleigh 分布
					\def
						$
							f_X(x) = \/{x}{σ^2} e^{\/{-x^2}{2 σ^2}}
						$
				* $Γ$分布
					\def
						$
							f(x) = \{\mb \/{1}{β^α Γ(α)} x^{a^{-1}} e^{-x / β} &\qu x \in (0, +∞) \\ 0 &\qu x \in (-∞, 0] \me\right.
						$
					\Property
						-
							$
								\E(x) = α β
								D(x) = α β^2
							$
						- 
							当$α=1$时, $Γ$分布退化为指数分布;
							当$α=n/2, β=1/2$时, $Γ$分布退化为$\chi^2$分布.
					* 指数分布
						\def
							$
								f(x) = \{\mb λ e^{-λ x} &\qu x \in (0, +∞) \\ 0 &\qu x \in (-∞, 0] \me\right.
								F(x) = \{\mb λ 1 - e^{-λ x} &\qu x \in (0, +∞) \\ 0 &\qu x \in (-∞, 0] \me\right.
							$
						\Property
							$
								\E(x) = θ
								D(x) = θ^2
							$
					* $\chi^2$分布
						\def
							$\/{1}{2^{n / 2} Γ(n / 2)} x^{n / 2 - 1} e^{-x / 2}$
						\Property
							$
								\E(x) = n
								D(x) = 2 n
							$
	* 随机过程
